\documentclass[11pt]{article}
\usepackage[english]{babel}
\usepackage[utf8x]{inputenc}
\usepackage[T1]{fontenc}
\usepackage[left=1.1in, right=.75in, top=.75in, bottom=.75in]{geometry}
\usepackage{graphicx}
\usepackage{dcolumn}
\usepackage{booktabs}
\usepackage{float}
\usepackage{subcaption}
\usepackage{array}
\usepackage{etoolbox}
\usepackage{setspace}
\usepackage{amsmath}
\usepackage{xcolor}
\usepackage{fancyhdr}
\usepackage{verbatim}
\usepackage{hyperref}
\usepackage{sectsty}
\usepackage{changepage}
\usepackage{longtable}
\usepackage{tikz}

\newcommand{\myarrow}[1][]{%
  \begin{tikzpicture}[#1]%
    \draw (0,0.7ex) -- (0,0) -- (0.75em,0);
    \draw (0.55em,0.2em) -- (0.75em,0) -- (0.55em,-0.2em);
  \end{tikzpicture}%
}

\newcommand{\PreserveBackslash}[1]{\let\temp=\\#1\let\\=\temp}
\newcolumntype{C}[1]{>{\PreserveBackslash\centering}p{#1}}
\newcolumntype{R}[1]{>{\PreserveBackslash\raggedleft}p{#1}}
\newcolumntype{L}[1]{>{\PreserveBackslash\raggedright}p{#1}}
\newcommand{\sym}[1]{\rlap{#1}}
\newcolumntype{d}[1]{D{.esesttabn t}{.}{#1}}

\newcommand\mc[1]{\multicolumn{1}{c}{#1}}% <---
\interfootnotelinepenalty=10000
\setlength{\parindent}{0em}
% \onehalfspacing
\singlespacing




\hypersetup{
    colorlinks = true,
    citecolor = blue,
    urlcolor  = blue,
}

\pagenumbering{gobble}
\sectionfont{\fontsize{12}{15}\selectfont}

\usepackage[pages=some]{background}
\backgroundsetup{
    scale=.18,
    color=black,
    opacity=1,
    angle=0,
    position=current page.north,
    vshift=-210mm,
     hshift=280mm,
    contents={%
        \includegraphics[width=\paperwidth]{ucsb_seal_navy.png}
    }
}


\hyphenpenalty=10000
\hbadness=10000

\begin{document}


\BgThispage
\noindent % Ensures no indentation at the beginning of the line
\begin{minipage}[t]{0.48\textwidth}
    \raggedright
    \hspace{-2em}{\LARGE \textbf{\textsc{Toshio Ferrazares}}} \vskip .5em
    Department of Economics\\
    University of California, Santa Barbara\\
    \texttt{ferrazares.github.io} \\
    \texttt{ferrazares@ucsb.edu} \\
    Citizenship: United States
\end{minipage}%
\hfill % Horizontal fill to push the next minipage to the right
\begin{minipage}[t]{0.48\textwidth}
    \raggedleft
    % \raisebox{-\height}{\includegraphics[width=7em]{ucsb_seal_navy.png}} % Adjust the image position
\end{minipage}




\vskip 2em
\hspace{-2em}\textbf{EDUCATION} \vskip .5em

UC Santa Barbara

\quad  Ph.D. Economics, 2025 (expected)

\quad  Field(s): \emph{Labor Economics}, \emph{Public Economics}

% \quad Dissertation: Essays in the Economics of Policing and Public Safety

\vskip 1em

San Diego State University

\quad  M.A., Economics, 2019

\quad  B.A., Economics, Minor in Mathematics, 2017


\vskip 2em
\hspace{-2em}\textbf{JOB MARKET PAPER}\vskip .5em
"\href{https://ferrazares.github.io/assets/FERRAZARESjmp.pdf}{Shift Structure and Cognitive Depletion: Evidence from Police Officers}".

\vskip .5em
   \hspace*{2em}\textbullet\quad Awarded "Best Paper Award" at Southern California Applied Economics Conference

\vskip .5em
\begin{adjustwidth}{3em}{1em}
    {\small    \textbf{Abstract}:
        Decision-making, risk-taking, and situational awareness are all important factors for effective and equitable policing.
        However, these factors can also be affected by fatigue, overwork, and cognitive stress, which can accumulate as police officers continue to work.
        This paper studies how working consecutive days affects police officer outcomes and activity using rich data from the Chicago Police Department.
        To overcome the endogenous selection of working days, I take advantage of a unique shift structure where working days are predetermined and based on fixed groupings.
        This is combined with a two-way fixed effects design that leverages within-officer variation across different working days.
        I find that as officers work more consecutive days, they use more force, make more judgement-based discretionary arrests, and are more likely to be injured.
        These increases occur despite a decline in proactive policing activities.
        Officers make fewer arrests, conduct fewer stops, issue fewer citations and tickets, and spend less time actively patrolling as their workdays accumulate.
        The divergence between use-of-force and policing activity is not driven by changes in arrest types, shift assignments, or officer roles, instead, officers are changing their behavior as they work more days.
    }
\end{adjustwidth}

\vskip 1em
\hspace{-2em}\textbf{PUBLICATIONS}\vskip .5em

"\href{https://www.sciencedirect.com/science/article/pii/S0094119023000086}{Monitoring Police with Body-Worn Cameras: Evidence from Chicago}", \emph{Journal of Urban Economics}, 2024.
\vskip .5em
   \hspace*{2em}\textbullet\quad Awarded "Best Second Year Paper Award" at UC Santa Barbara Economics

\vskip .5em
\begin{adjustwidth}{3em}{1em}
    {\small    \textbf{Abstract}:
        Using data from the Chicago Police Department on complaints filed by civilians and reports of force filed by officers, this paper estimates the effect of body-worn cameras (BWCs) of officer and civilian behavior. Using a two-way fixed effects design, I find BWCs are associated with a 29\% reduction in use-of-force complaints, driven by white officer-black civilian complaints. Additionally, I find a 34\% reduction in officers reporting striking civilians and a large though less significant reduction in officer firearm usage, potential mechanisms for the reduction in complaints. Importantly, I find no change in officer injury or force from civilians. However, I find evidence of de-policing as officers make fewer drug-related arrests following BWC adoption.
    }
\end{adjustwidth}


\vskip 2em
\hspace{-2em}\textbf{WORK-IN-PROGRESS}\vskip .5em
\begin{enumerate}




    \item "\href{https://michaeltopper.netlify.app/media/jmp_michael_topper.pdf}{The Unintended Consequences of Policing Technology: Evidence from ShotSpotter}", with Michael~Topper, \emph{Under Review}.
        \vskip .5em
   \hspace*{2em}\textbullet\quad Media Coverage:
\href{https://www.economist.com/united-states/2023/12/27/americas-new-policing-tech-isnt-cutting-crime?giftId=77056dba-6af1-40be-af4c-0f6e5baf028b}{The Economist},
\href{https://www.chicagotribune.com/2024/03/03/with-shotspotter-staying-in-chicago-for-the-time-being-dispute-continues-over-the-systems-usefulness/}{The Chicago Tribune},
\href{https://stateline.org/2024/02/27/chicago-is-the-latest-city-rethinking-disputed-technology-that-listens-for-gunshots/}{Stateline}


\vskip 1em
          \begin{adjustwidth}{3em}{1em}
              {\small    \textbf{Abstract}:
                  Technology is integral to police departments, automating officer tasks, but inherently changing their time allocation.
                  We investigate this by studying ShotSpotter, a technology that automates gunfire detection.
                  Following a detection, officers are dispatched to the scene, thereby reallocating their time.
                  We leverage this shock to officers’ time allocation using the rollout of ShotSpotter across Chicago police districts to study the effects on 911 call response.
                  We find substantial consequences, officers are dispatched to calls slower (23\%), arrive on-scene later (13\%), and the probability of arrest is decreased 9\%.
                  Consequently, police departments must evaluate their resource capacities prior to implementing technologies.
              }
          \end{adjustwidth}\vskip 1em

    \item "\href{https://www.nber.org/system/files/working_papers/w28763/w28763.pdf}{Have U.S. Gun Buyback Programs Misfired?}" with Joseph~J.~Sabia and D.~Mark~Anderson, Revisions Requested at \emph{Journal of Policy Analysis and Management}. NBER Working Paper \#28763.
        \vskip .5em
   \hspace*{2em}\textbullet\quad Media Coverage:
\href{https://www.nbcphiladelphia.com/news/local/broke-in-philly/with-gun-buybacks-philadelphia-fighting-uphill-battle/2823158/}{NBC Philadelphia},
\href{https://reason.com/2021/05/17/gun-buybacks-dont-seem-to-significantly-lower-gun-crimes-or-gun-deaths/}{Reason},
\href{https://www.cnn.com/2022/04/16/us/chicago-gun-buybacks/index.html}{CNN},
\href{https://www.houstonchronicle.com/news/houston-texas/crime/article/houston-gun-buyback-crime-impact-17312534.php}{Houston Chronicle},
\href{https://www.wvtf.org/news/2023-03-13/charlottesville-considers-gun-buyback-program-is-it-worthwhile}{Virginia Public Radio},
\href{https://www.cato.org/research-briefs-economic-policy/have-us-gun-buyback-programs-misfired#}{CATO Institute},
\href{https://stateline.org/2022/08/30/gun-buybacks-are-popular-but-are-they-effective/}{Stateline},


\vskip 1em
          \begin{adjustwidth}{3em}{1em}
              {\small    \textbf{Abstract}:
                  Gun buyback programs (GBPs), which use public funds to purchase civilians' privately-owned firearms, aim to reduce gun violence.
                  However, next to nothing is known about their effects on firearm-related crime or deaths.
                  Using data from the National Incident Based Reporting System, we find no evidence that GBPs reduce gun crime.
                  Given our estimated null findings, with 95 percent confidence, we can rule out decreases in firearm-related crime of greater than 1.1 percent during the year following a buyback.
                  Using data from the National Vital Statistics System, we also find no evidence that GBPs reduce suicides or homicides where a firearm was involved. These results call into question the efficacy of city gun buyback programs in their current form.
              }
          \end{adjustwidth}\vskip 1em

    \item “Drinking Water Contaminants and Infant Health”, with Katherine~Grooms, Heather~Royer, and Kevin~Schnepel.

\end{enumerate}


\vskip 2em
\hspace{-2em}\textbf{REFERENCES} \vskip .5em


\begin{minipage}[t]{0.34\textwidth}
    Professor Heather Royer \\
    UC Santa Barbara \\
    Committee Chair \\
    \texttt{heather.royer@ucsb.edu}
\end{minipage}
\begin{minipage}[t]{0.3\textwidth}
    Professor Peter Kuhn \\
    UC Santa Barbara \\
    Committee Member \\
    \texttt{peter.kuhn@ucsb.edu}
\end{minipage}
\begin{minipage}[t]{0.3\textwidth}
    Professor Kevin Schnepel \\
    Simon Fraser University \\
    Committee Member \\
    \texttt{kevin\_schnepel@sfu.ca}
\end{minipage}
\vskip 1em

\begin{minipage}[t]{0.34\textwidth}
    Placement Director \\
    Professor Daniel Martin \\
    UC Santa Barbara \\
    \texttt{danielmartin@ucsb.edu}
\end{minipage}

\clearpage
\vskip 2em
\hspace{-2em}\textbf{HONORS AND AWARDS}\vskip .5em
\vskip .5em

\begin{tabular}{ ll }
    2024       & Dissertation Summer Fellowship (UCSB)              \\
    2023       & Gretler Research Quarter Fellowship (UCSB)         \\
    2021       & Best 2nd Year Paper Award (UCSB)                   \\
    2019       & M.C. Madhavan Prize for Outstanding Student (SDSU) \\
    2017, 2018 & McCuen Fellowship (SDSU)                           \\
    2017, 2018 & Center for Public Economics Scholarship (SDSU)     \\
    2017, 2018 & The Weintraub Paper Award (SDSU)
\end{tabular}

\vskip 2em
\hspace{-2em}\textbf{EMPLOYMENT AND EXPERIENCE}\vskip .5em
\vskip .5em

\begin{tabular}{ll}
    2023-24      & Teaching Assistant, Labor Economics, Personnel Economics II        \\
    2022-23      & Teaching Assistant, Data Wrangling for Economists                  \\
    2021-Present & Research Assistant, Professor Heather Royer                        \\
    2020-2021    & Teaching Assistant, Intro to Macroeconomics                        \\
    2019-2020    & Teaching Assistant, Intro to Microeconomics                        \\
    2019-Present & Doctoral Affiliate, Center for Health Economics and Policy Studies \\
    2017-2019    & Research Assistant, Professor Joseph Sabia                         \\
\end{tabular}

\vskip 1em
Technical Expertise: R, Stata, Python, ArcGIS, MATLAB

\vskip 1em
Referee Service: \textit{
    Journal of Labor Economics
}

\vskip 2em
\hspace{-2em}\textbf{SEMINARS AND CONFERENCES}\vskip .5em
\vskip .5em

Southern California Applied Economics Conference (UC Irvine), 2024\\
\hspace*{2em}\myarrow\emph{Best Paper Award} for \emph{Shift Structure and Cognitive Depletion: Evidence from Police Officers}\\
All California Labor Economics Conference (UC Los Angeles), 2024\\
WEAI Graduate Student Workshop, 2024\\
Western Economic Association International, 2023\\
All California Labor Economics Conference (UC Santa Barbara), 2023\\
Eastern Economics Association, 2019\\
Western Economic Association International, 2019\\
Association for Public Policy Analysis and Management, 2019\\
Eastern Economics Association, 2018






\end{document}




